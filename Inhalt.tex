\documentclass[12pt,a4paper,oneside,titlepage]{report}

%\textwidth=15cm
%\hoffset=-1 cm
\usepackage{ngerman}
\usepackage[latin1]{inputenc}
\usepackage{ae}
\usepackage{graphicx}
\usepackage{amsmath}
\usepackage{amsthm}
\usepackage{dsfont}
\usepackage{listings}
\lstset{language=C}
%\usepackage{amsfonts,textcomp}
%\usepackage{calc}

\usepackage{makeidx,showidx} \makeindex
\usepackage{multirow}
\usepackage{tabularx}
\usepackage{subfigure}
\usepackage{epic}
\usepackage{longtable}
\usepackage{rotating}
\usepackage{float}
\usepackage{fancyvrb}
%\usepackage{glossary}
%URL Farb und Link Anpassung
%\usepackage{fancyhdr}
\usepackage{color}
\usepackage{hyperref}
\usepackage{lscape}
\definecolor{black}{rgb}{0,0,0}
\definecolor{darkblue}{rgb}{0,0,0.5}
\hypersetup{colorlinks, linkcolor=black, urlcolor=darkblue, citecolor=black, pdftitle={Implementierung und Optimierung von Algorithmen zur Audiosignal-Klassifikation auf einer heterogenen Prozessorarchitektur}, pdfauthor={Kristian Wolpers}, pdfsubject={Masterarbeit, Juli 2013}}


\renewcommand{\arraystretch}{1.5}
\newcommand\textsubscript[1]{\ensuremath{{}_{\text{#1}}}}

\addtolength{\headheight}{0,3cm}
\addtolength{\headsep}{-0,3cm}
\setlength{\parindent}{0pt}
\setlength{\parskip}{12pt plus 6pt minus 3pt}
\setlength{\partopsep}{0pt}
\addtolength{\topmargin}{-0,6cm}
\addtolength{\oddsidemargin}{-0,6cm}
\addtolength{\textwidth}{2cm}
\addtolength{\textheight}{2cm}

\restylefloat{figure}
\IfFileExists{url.sty}{\usepackage{url}}

\usepackage{fancyhdr}
\pagestyle{fancy}


\renewcommand{\headrulewidth}{0.4pt}
\renewcommand{\sectionmark}[1]{\markright{ Kapitel  \thesection\ \it #1}}
\renewcommand{\chaptermark}[1]{\markright{ Kapitel  \thechapter\ \it #1}}

\lhead{\nouppercase{\rightmark}}
\rhead{\thepage}
\cfoot{}

% Damit auch die Titelseiten fancy sind!
\makeatletter
\let\ps@plain=\ps@empty
\makeatother

%Fourier-Korrespondenz link  *-o

\newcommand{\ftranl}{\bullet\!\!-\!\!\circ}

%

%Fourier-Korrespondenz rechts  o-*

\newcommand{\ftranr}{\circ\!\!-\!\!\bullet} 

%

%Fourier-Korrespondenz unten o

%                            |

%                            *

\newcommand{\ftranb}{{\setlength{\unitlength}{12pt}\begin{picture}(0.5,1.6)\put(0,1){$\circ$}\put(0.186,0.186){\rule[2.5pt]{0.6pt}{7.5pt}}\put(0,0){$\bullet$}\end{picture}}}

%

%Fourier-Korrespondenz oben  *

%                            |

%                            o

\newcommand{\ftrant}{{\setlength{\unitlength}{12pt}\begin{picture}(0.5,1.6)\put(0,1){$\bullet$}\put(0.186,0.186){\rule[2.5pt]{0.6pt}{7.5pt}}\put(0,0){$\circ$}\end{picture}}}

\begin{document}

% Listing Package definieren
\lstset{
framexleftmargin=6mm,
framexrightmargin=0mm,
framextopmargin=2mm,
framexbottommargin=2mm,
frame=lines,
basicstyle=\scriptsize, % \footnotesize,
keywordstyle=\bfseries,
identifierstyle=,
commentstyle=\color{white},
%stringstyle=\ttfamily,
showstringspaces=false,
numbers=left,
numberstyle=\tiny,
xleftmargin=18pt
%morekeywords={MV,MACI_U8,MACPLZI_U8,SUB_16,PERMREG0_8,ABSADDI_8,CLIP_U8,SMVI,SUBICS_8,SUBCR_8,MVCR_8,V0R10,V1R23,V1R7,V0R28,V1R4,V1R0,V1R1,V0CONDSEL}
}

\pagenumbering{alph}

%\include{Verschwiegenheit}
\pagenumbering{Roman}                   % R�mische Seitennummern �ber den Vorspann
\setcounter{secnumdepth}{4}             % Nur section und subsection numerieren
%Inhaltsverzeichnis erstellen                                        
{\setlength{\parskip}{2pt}
\pdfbookmark[1]{Inhaltsverzeichnis}{toc}
\setcounter{tocdepth}{4}
\tableofcontents}                   
%Abbildungsverzeichnis erstellen
{\setlength{\parskip}{2pt}
\listoffigures}
\newpage
\pagenumbering{arabic}

\sloppy                            % be a bit more tolerant
\hbadness2000 \vbadness2000

% Disable single lines at the start of a paragraph (Schusterjungen)
\clubpenalty = 10000
%
% Disable single lines at the end of a paragraph (Hurenkinder)
\widowpenalty = 10000 \displaywidowpenalty = 10000

\chapter{Einleitung}

\chapter{Grundlagen}
\section{Musikklassifikation}
\subsection{Extraktion}
\subsection{Prozessierung}
\subsection{Klassifikation}
\section{Algorithmen}
\subsection{Notation von Ausdr�cken und Variablen}
\subsection{Features des Zeitbreichs}
\subsubsection{Root Mean Square (RMS)}
\subsubsection{Low Energy}
\subsubsection{Zero Crossing Rate}
\subsection{Features des Frequenzbereichs}
\subsubsection{Fast Fourier Transformation (FFT)}
\paragraph{FFT mit Zeitzerlegung (DIT)}\label{ph:dit}$\;$ \\
\paragraph{FFT mit Frequenzzerlegung (DIF)}\label{ph:dif}$\;$ \\
\paragraph{Bitumkehrungordnung (bit-reversed order)}\label{ph:bit}$\;$ \\
\subsubsection{Amplitude of Spectrum}
\subsubsection{Chroma Vektor}
\subsubsection{Amplitude Of Maximum In Chromagram}
\subsubsection{Mel Frequency Cepstral Coefficients (MFCC)}
\subsubsection{Normalized Audio Spectrum Envelope}
\subsubsection{Octave Spectral Contrast}
\subsubsection{Spectral Centroid}
\subsubsection{Spectral Crest Factor}
\subsubsection{Spectral Flux}
\subsubsection{Spectral Rolloff}
\subsubsection{Sub-Band Energy Ratio}
\section{Featuresets}
\subsection{Featureset 1 (FSet1)}
\subsection{Featureset 2 (FSet2)}
\subsection{Featureset 3 (FSet3)}
\subsection{Featureset 4 (FSet4)}

\chapter{Board}
\section{C674x}
\begin{itemize}
\item Hardware
\item evtl. Compileroptimierungsstufen (c6run)
\item Bsp: Opt. ZCR, SPLOOP SC
\item opt. Bibliotheken anrei�en
\end{itemize}
\chapter{Optimierungen}
 Optimierungen nur an FSET1 vorstellen, andere �hnlich, FSET3 eventuell noch gesondert behandeln

\section{DSP}

\begin{enumerate}
\item Laufzeitmessung von reinen Compileroptimierungen
\item Bottlenecks: AoS und FFT
\begin{itemize}
\item Aos: SQRT und DIV
\item FFT: BR+Butterfly+NV
\item DSPLIB und MATHLIB als L�sung vorstellen
\begin{itemize}
\item MATHLIB ins Details gehen: DIV und SQRT auf Alg- und Hardwareebene
\item DSPLIB: FFT-Struktur beschreiben inkl. radix4 (evtl. Graph, ASM, Pipeline)
\end{itemize}
\item Bewertung durch Laufzeitmessung
\end{itemize}

\end{enumerate}

\chapter{Evaluation}

\chapter{Zusammenfassung und Ausblick}


%bibtex Referenzen in bib datei
\bibliographystyle{Macros/unsrtdin}
%\bibliographystyle{unsrt}
%Literaturverzeichnis ins Inhaltsverzeichnis
\addcontentsline{toc}{chapter}{Literaturverzeichnis}

\appendix %%Anhang
\renewcommand{\chaptermark}[1]{\markright{Appendix \thechapter\ \it #1}}
\renewcommand{\sectionmark}[1]{\markright{Appendix \thechapter\ \it #1}}
\renewcommand{\appendixname}{Anhang}
\renewcommand{\tablename}{Table}
\renewcommand{\figurename}{Figure}



\end{document}