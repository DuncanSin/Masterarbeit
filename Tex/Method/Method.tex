\chapter{Implementierung \& Optimierung}
\label{ch:opt}
\rm

In diesem Kapitel werden die Implementierung und die Optimierung des Referenz-Programm-Codes auf dem ARM Cortex A8 und dem C674x DSPs beschrieben.\\
Hierf�r wir in \textbf{Abschnitt \ref{sec:str}} als erstes die Optimierungsstrategie vorgestellt, nach der die Optimierungen in den folgenden Abschnitten durchgef�hrt werden. Anschlie�end werden in \textbf{Abschnitt \ref{sec:optarm}} die Implementierung und Optimierung des Referenz-Programm-Codes auf dem ARM Cortex A8 und in \textbf{Abschnitt \ref{sec:optdsp}} die auf dem C674x DSP vorgestellt.

\section{Optimierungsstrategie}\label{sec:str}
Das Ziel der in den folgenden Abschnitten vorgestellten Optimierungen ist die Verbesserung der Laufzeit der Musikklassifikation. Hierf�r wird angestrebt, die von den Prozessoren zur Verf�gung gestellten Rechenressourcen effizient auszunutzen.\\
F�r die Berechnung dieser Effizienz sind mehrere G�tema�e vorhanden, unter anderem Werte f�r \textit{Instruktionen pro Takt} (engl. Instruction per Cycle oder kurz IPC), \textit{Takte pro Instruktion} (engl. Cycles per Instruktion oder kurz CPI), die Messung von \textit{Cache Hits und Misses} und die \textit{Laufzeit} des Programms. Aufgrund der von Texas Instrument zur Verf�gung gestellten Werkzeuge ist eine Messung von IPC/CPI-Werten oder Cache Hits/Misses nicht ohne weiteres m�glich, so dass in dieser Arbeit auf die Messung der Laufzeit zur Bewertung der Effizienz zur�ckgegriffen wird.\\
Um bei der Optimierung den gr��ten Gewinn zu erreichen, werden gezielt die Programmteile optimiert, die den gr��ten Anteil an der Verarbeitungszeit des Musikklassifikators besitzen. Hierf�r wird als erstes aus den drei Schritten der Musikklassifikation derjenige identifiziert, der am Laufzeit-intensivsten ist. Als n�chstes werden aus diesem Schritt ebenfalls jene Algorithmen identifiziert und optimiert, die den gr��ten Anteil an der Verarbeitungszeit dieses Schrittes einnehmen. Auf diese Weise wird versucht, mit gezielten Optimierungen gro�e Verbesserungen an der Gesamtverarbeitungszeit zu erreichen. 
Als n�chstes muss entschieden werden, wie Optimierungspotenziale identifiziert werden.\\
Sowohl die Untersuchungen auf dem ARM Cortex-A8 als auch die auf dem C674x DSP werden hierf�r in drei Schritten durchgef�hrt:

\begin{enumerate}
\item Laufzeitanalyse des Programmteils zur Identifikation zeitkritischer Algorithmen oder Codesegmente
\item Optimierung der gefundenen Algorithmen oder Codesegmente
\item Bewertung der durchgef�hrten Optimierungen
\end{enumerate}

Hierbei handelt es sich um einen iterativen Prozess. Das hei�t, dass nach einer durchgef�hrten Optimierung eine weitere Laufzeitmessung durchgef�hrt wird. Dieses geschieht zum einen f�r die Bewertung einer durchgef�hrten Optimierung und zum anderen, um eventuell neue zeitkritische Algorithmen zu identifizieren.
