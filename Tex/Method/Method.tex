\chapter{Methodik der Arbeit}
\label{ch:method}
\rm

In diesem Kapitel soll die Methodik beschrieben werden, mit der die Untersuchungen in den folgenden Kapiteln durchgef�hrt wurden.\\
Sowohl die Untersuchungen auf dem ARM Cortex\texttrademark-A8, als auch die auf dem DSP wurden in drei Schritten durchgef�hrt:

\begin{enumerate}
	\item Laufzeitanalyse des Programms durch Profiling (\textbf{\ref{sec:prof}})
	\item Identifikation der Bottlenecks aus den Ergebnissen der Laufzeitanalyse
	\item Optimierung der gefundenen Bottlenecks (\textbf{\ref{sec:opt}})
\end{enumerate}

Im Folgenden sollen die Begriffe der einzelnen Schritte n�her erkl�rt werden.

\section{Profiling}\label{sec:prof}
Der Begriff Profiling beschreibt Methoden, mit denen das Verhalten von Applikationen auf Systemebene analysiert werden k�nnen, hierzu z�hlen Analysen der Laufzeit, der Schreib- und Lesezugriffe und auch die Verh�ltnisse von Instruktionen und Takten zueinander. Im Folgenden werden nun Beispielhaft die Profilingmethoden der Laufzeitmessung, der richtigen Vorhersagen des Vorladens in den Cache (sog. Cache Hits), sowie die Betrachtungen des Verh�ltnisses von Takt und Instruktionen (Instructions per Cycle (IPC) und Cycle per Instruction (CPI)).

\subsection{Laufzeitanalyse}\label{subsec:time}

Bei der Laufzeitanalyse wird die reale Ausf�hrungszeit eines Programmes oder Teilen eines Programmes auf einem gegebenen System betrachtet. Dieses geschieht entweder durch eine externe Uhr, z.B. eine Armbanduhr oder durch eine in dem System integrierte Uhr durch das Auslesend er entsprechenden Register. Sollte eine interne Uhr verwendet werden, muss beachtet werden, dass in den entsprechenden Registern der momentane Takt gespeichert ist, also f�r eine Zeitangabe noch durch die Taktrate des Systems geteilt werden muss. Sollte die zumessende Laufzeit sehr gering sein, empfiehlt sich die Nutzung einer internen Uhr.\\
Es ist leicht ersichtlich, dass der gemessene Wert zwischen 0 und $\infty$ liegt und dass das Ergebnis umso besser ist, desto n�her es an der 0 liegt.

\subsection{Cache Hits}\label{subsec:cache}

\subsection{IPC und CPI}\label{subsec:ipccpi}

Bei \textit{Instructions per Cycle} (IPC) und \textit{Cycles per Instruction} (CPI) werden die ausgef�hrten Instruktionen mit den ben�tigen Takten ins Verh�ltnis gesetzt um damit eine Aussage �ber die Effizienz des Codes zu geben.\\
F�r die Berechnung des IPC wird die Anzahl der f�r die Ausf�hrung ben�tigen Instruktionen durch die Anzahl der ben�tigen Takte geteilt. Da eine Instruktion mindestens einen Takt und maximal n Takte ben�tigt ergibt sich aus diesem Verh�ltnis ein Wert zwischen 0 und 1. Umso n�her der resultierende Wert ein 1 liegt, desto effizienter ist der Code, ein Wert von 1 w�rde bedeuten, dass in jedem Takt eine Instruktion ausgef�hrt wird.\\
Die Berechnung des CPI ist etwas komplexer. Hierbei werden die Anzahl von Instruktionstypen im Code (IIC), deren ben�tigte Takte (CCI) und die Gesammtzahl der ben�tigen Instruktionen (IC) miteinander ins Verh�ltnis gesetzt (vgl. \textbf{Formel \ref{eqn:cpi}}).

\begin{equation}\label{eqn:cpi}
 CPI~=~\frac{\sum\left(IIC \cdot CCI\right)}{IC}
\end{equation}

Hierbei spielt neben der ben�tigen Takte pro Instruktionstyp auch noch die vorliegende Pipelinetiefe eine Rolle. Nimmt man an man h�tte eine 5 Stufen Pipeline und einen Instruktionstyp der 5 Takte ben�tigt, w�re es so trotzdem m�glich pro Takt eine Instruktion dieses Typs auszuf�hren und somit einen CPI-Wert von 1 zu erreichen. Daraus ergibt sich, dass der CPI-Wert zwischen 0 und $\infty$ liegt, wobei ein Code effizienter ist, umso n�her der CPI-Wert an 0 liegt.

\section{Optimierung}\label{sec:opt}

Unter Optimierung versteht man die Verbesserung einer Sache hinsichtlich auf ein zu erreichendes Ziel.\\
In der Softwareentwicklung sind solche Ziele in Normalfall Laufzeit oder Speicherbedarf. Nimmt man noch die Hardwareentwicklung hinzu, ergeben sich als weitere Ziele noch der Platzbedarf, den ein zu entwickelndes System, in den mei�ten F�llen ein Chip, nach der Fertigung ben�tig, sowie der Leistungsverbrauch, den das dieses System aufweist .\\
In dieser Arbeit sollen Optimierungen prim�r hinsichtlich der Laufzeit stattfinden. Als zweiter Aspekt soll aber auch der Leistungsverbrauch der einzelnen Optimierungsstufen betrachtet werden, da eine Betrachtung hinsichtlich der Einsetzbarkeit auf mobilen Endger�ten angestrebt wird.