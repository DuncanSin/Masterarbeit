\chapter{Implementierung \& Optimierung}
\label{ch:opt}
\rm

In diesem Kapitel werden die Implementierung und die Optimierung des Referenz-Programm-Codes auf dem ARM Cortex A8 und dem C674x DSPs beschrieben.\\
Hierf�r wir in \textbf{Abschnitt \ref{sec:str}} als erstes die Optimierungsstrategie vorgestellt nach der die Optimierungen in den folgenden Abschnitten durchgef�hrt werden. Anschlie�end werden in \textbf{Abschnitt \ref{sec:optarm}} die Implementierung und Optimierung des Referenz-Programm-Codes auf dem ARM Cortex A8 und in \textbf{Abschnitt \ref{sec:optdsp}} die auf dem C674x DSP vorgestellt.

\section{Optimierungsstrategie}\label{sec:str}
Das Ziel der in den folgenden Abschnitten vorgestellten Optimierungen ist die Verbesserung der Laufzeit des Musikklassifikator. Hierf�r wird angestrebt, die von den Prozessoren zur Verf�gung gestellten Rechenressourcen effizient auszunutzen.\\
F�r die Berechnung dieser Effizienz sind mehrere G�tema�e vorhanden, unter anderem Werte f�r \textit{Instruktionen pro Takt} (engl. Instruction per Cycle oder kurz IPC), \textit{Takte pro Instruktion} (engl. Cycles per Instruktion oder kurz CPI), die Messung von \textit{Cache Hits und Misses} und die \textit{Laufzeit} des Programms. Auf Grund der von Texas Instrument zur Verf�gung gestellten Werkzeuge ist eine Messung von IPC/CPI-Werten oder Cache Hits/Misses nicht ohne weiteres M�glich, so dass sich in dieser Arbeit auf die Messung der Laufzeit f�r die Bewertung der Effizienz beschr�nkt werden soll.\\
Als n�chstes muss entschieden werden, wie Optimierungspotenziale identifiziert werden sollen. Die Herausforderung hierbei liegt darin, dass eine Vielzahl von Algorithmen vorhanden sind. Es w�re m�glich diese nach einer zu w�hlenden Reihenfolge sukzessiv zu Optimieren. Dieses w�rde aber viel Zeit in Anspruch nehmen und es w�rde eventuell mehr Zeit in Optimierungen investiert die keinen gro�en Anteil an der Gesamtlaufzeit besitzen. Daher wird sich die Optimierung innerhalb dieser Arbeit auf die Optimierung von zeitkritischen Stellen beschr�nken. Hierf�r wird als erstes der Schritt der Musikklassifikation identifiziert werden, der den gr��ten Anteil an der Gesamtlaufzeit besitzt. Nach dieser Identifizierung wird die Optimierung innerhalb dieses Schritts ebenfalls auf jene Algorithmen beschr�nkt, die ihrerseits den gr��ten Anteil an der Laufzeit dieser Phase besitzen. Auf diese Weise wird versucht mit der Optimierung von einzelnen Algorithmen gro�e Verbesserungen an der Gesamtlaufzeit des Musikklassifikators zu erreichen.\\
Sowohl die Untersuchungen auf dem ARM Cortex A8, als auch die auf dem C674x DSP werden hierf�r in drei Schritten durchgef�hrt:

\begin{enumerate}
\item Laufzeitanalyse des Programmteils zur Identifikation zeitkritischer Algorithmen oder Codesegmente
\item Optimierung der gefundenen Algorithmen oder Codesegmente
\item Bewertung der durchgef�hrten Optimierungen
\end{enumerate}

Hierbei handelt es sich um einen iterativen Prozess, das hei�t, dass nach einer durchgef�hrten Optimierung eine weitere Laufzeitmessung durchgef�hrt wird. Dieses geschieht zum einen f�r die Bewertung einer durchgef�hrten Optimierung und zum anderen um eventuell neue zeitkritische Algorithmen zu identifizieren.
