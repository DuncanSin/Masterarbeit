\chapter{Evaluation}
\label{ch:results}
\rm

\section{Einf�hrung}

\begin{itemize}
\item Einf�hren der Bewertungsmetriken:
\begin{itemize}
\item Extraktionsrate: $\frac{Fenster}{Zeit}$
\item Effizienz: $\frac{Geschwindigkeit}{Ennergieverbrauch}$
\end{itemize}
\end{itemize}

\section{Evaluation der ARM-Optimierung}

\subsection{Qualit�t der Optimierung}

\begin{itemize}
\item Extraktionsraten der FeatureSets
\item Effizienz der FeatureSets
\end{itemize}

\subsection{Weitere Optimierungspotenziale}

\begin{itemize}
\item Sortierung (76\%) bei OSC
\item DCT (37,6\%) bei MFCC, Logarithmus (38\%) nicht viel machbar
\item Signum (90\%) bei ZCR nicht viel machbar
\item MAG von Hand optimieren (Intrinsics oder ASM)
\item CV nicht viel machbar, da Einordnungsteil zu unvorhersagbar
\item HW mit ASM, spart eventuell Lade- und Speicherbefehle (Durchg�ngiger Datenpfad)
\item andere Eventuell auch parallelisierbar mit NEOn, aber irrelevant, weil zu geringe Laufzeiten
\end{itemize}

\section{Evaluation der DSP-Optimierung}

\subsection{Qualit�t der Optimierung}
\begin{itemize}
\item Extraktionsraten der FeatureSets
\item Effizienz der FeatureSets
\end{itemize}

\subsection{Weitere Optimierungspotenziale}

\begin{itemize}
\item Sortierung (55,5\%) bei OSC
\item Akkumulation (73,6\%) mit ASM und DCT (23,6\%) bei MFCC
\end{itemize}

\section{Evaluation heterogenes System}

\subsection{Einschr�nkungen}

\begin{itemize}
\item Taktzahlgrenzen der Prozessoren => H�herer Takt = H�herer Energieverbrauch => evtl. ineffizienter
\item Kommunikation, bei zu gro�en Lieder auch iteratives Schreiben in den Shared-Speicher => Kostet Zeit
\end{itemize}

\subsection{Gegen�berstellung ARM und ARM+DSP}

\begin{itemize}
\item Laufzeitdiskusion
\item Leistungsverbrauch
\item Effizienz
\end{itemize}

\section{Vorstellung ARM als heterogenes System}

$\;$ \\
$\;$ \\
$\;$ \\
$\;$ \\
$\;$ \\
$\;$ \\
$\;$ \\
$\;$ \\
$\;$ \\
$\;$ \\

\section{Fazit}

\begin{itemize}
\item ARM+DSP zu ineffizient
\begin{itemize}
\item Kommunikationszeit
\item Erh�hter Energiebedarf durch Zuschalten von DSP
\item DSP bringt keinen wirklichen Geschwindigkeitszuwachs
\end{itemize}
\item ARM+NEON effizienter, nach weiteren Optimierungen evtl. auch schneller
\end{itemize}