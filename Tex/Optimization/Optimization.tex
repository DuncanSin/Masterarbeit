\chapter{Optimierung des MusicClassifiators}
\label{ch:optimization}
\rm

In diesem Kapitel werden die Anfangsbedingungen, die Ansatzpunkte, die Konzepte der durchgef�hrten Optimierungen am Code des MusicClassificators gehen.\\
Hierzu werden in \textbf{Kapitel \ref{sec:start}} zuerst die Messungen der Laufzeit des Originalcodes vorgestellt, aus denen in \textbf{Kapitel \ref{sec:ansatz}} die wesentlichen
Bottlenecks und Optimierungspotenziale extrahiert werden sollen. \textbf{Kapitel \ref{sec:oparm}} befasst sich daraufhin mit dem Konzept und der Durchf�hrung der Optimierungen,
welche f�r eine rein ARM-seitiges Applikation unternommen wurden, bevor in \textbf{Kapitel \ref{sec:opdsp}} das Konzept einer Optimierung durch Einbinden des integrierten digitalen Signalprozessors (DSP)
vorgestellt wird.

\section{Anfangsbedingungen}\label{sec:start}

\section{Ansatzpunkte und Bottlenecks}\label{sec:ansatz}

\section{Optimierung des ARM-Codes}\label{sec:oparm}

\section{Optimierung durch Einbinden des DSP}\label{sec:opdsp}
 