
\section{Optimierung des ARM}
In diesem Kapitel werden die Optimierungen am Code beschrieben, der rein auf dem ARM Cortex-A8 ausgef�hrt wird. Hierf�r werden die in \textbf{Kapitel \ref{subsec:a8}} beschriebenen Hardwareemelemte ausgenutzt.\\
In \textbf{Abschnitt \ref{subsec:armtime}} wird daher als erstes die Laufzeitmessung des gesamten Programms (Extraktion, Prozessierung und Klassifikation) erl�utert, um einen Einstiegspunkt f�r die Optimierung zu extrahieren. In den darauffolgenden Abschnitten werden daraufhin die durchgef�hrten Optimierungen beschrieben. 
\subsection{Laufzeitmessung des Gesamtprogramms}\label{subsec:armtime}

\subsection{Laufzeitmessung der Extraktion}

\subsection{Ansatzpunkte und Bottlenecks}\label{sec:ansatz}


\subsubsection{FFT}


\subsection{Libav als Optimierung der FFT}\label{subsec:optFFT}
\subsubsection{Laufzeitmessung}
\subsubsection{Aufbau der FFT}
\subsubsection{Einbindung}

\subsection{Optimierung der Amplitude of Spectrum}\label{subsec:optAOS}
\subsubsection{Laufzeitmessung}
\subsubsection{Codeanpassung}

\subsection{Optimierung von MFCC}

\subsection{Optimierung der Zero Crossing Rate}



 