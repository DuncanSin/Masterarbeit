\chapter{Grundlagen}
\label{grund}
\rm

\section{Profiling}\label{sec:prof}
Der Begriff Profiling bescheibt Methoden, mit denen das Verhalten von Applikationen auf Systemebene analysiert werden k�nnen, hierzu z�hlen Analysen der Laufzeit, der Schreib- und Lesezugriffe und auch die Verh�ltnisse von Instuktionen und Takten zueinander. Im Folgenden werden nun Beispielhaft die Profilingmethoden der Laufzeitmessung, der richtigen Vorhersagen des Vorladens in den Cache (sog. Cache Hits), sowie die Betrachtungen des Verh�ltnisses von Takt und Instuktionen (Instructions per Cycle (IPC) und Cycle per Instruction (CPI)).

\subsection{Laufzeitanalyse}\label{subsec:time}

\subsection{Cache Hits}\label{subsec:cache}

\subsection{IPC und CPI}\label{subsec:ipccpi}

