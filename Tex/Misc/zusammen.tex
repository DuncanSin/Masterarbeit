\chapter{Zusammenfassung}
\label{Ch7}
\rm

Am Institut f�r Mikroelektronische Systeme der Leibniz Universit�t Hannover wird ein Demonstrationssystem f�r Echtzeit-Bild- und Video-Signalverarbeitung entwickelt. Das System ist FPGA-basiert und besitzt internen und externen Speicher sowie mehrere Bearbeitungseinheiten f�r verschiedene Bildverarbeitungsalgorithmen. Die internen Teil-Module sind �ber den Module Interconnect Bus (MIB) verbunden. F�r eine Kommunikation mit einem RISC oder Host-PC steht ein Data-Interface zur Verf�gung.

Aufgabe dieser Studienarbeit war es, eine Recheneinheit f�r den Labeling-Algorithmus zu entwerfen und diese als Processing-Element (PE) in das Gesamtsystem zu integrieren. Dabei greift die Labeling-Einheit auf die Ausgaben vorheriger PEs zu, die mit Filteroperationen das Eingangsbild vorverarbeitet und in ein Bin�rbild umgewandelt haben. Der Speicherort der Ein- und Ausgangsdaten stellt das DDR-RAM des Demonstrationssystems dar. F�r die  Labeling-Prozedur wurde eine Fenstererweiterung eingef�hrt und ein neues Verfahren zur Verarbeitung von Label-�quivalenzen entwickelt. Durch diese Ma�nahmen wurde eine deutliche Verk�rzung der Bearbeitungszeit erreicht. Als ein besonderes Feature ist eine Behandlung von Region of Interests (ROI) m�glich, welche nur einen geforderten Bildbereich bearbeitet und somit Rechenaufwand reduziert.

Das Modul liegt auf RTL- und Gatterebene als verifizierter VHDL-Quelltext vor, welcher f�r MIB-Busbreiten von 64 und 128 Bit synthetisiert werden kann. Eine Synthese mit der Software Synplify Pro 8.8.0.4 ergab f�r einen Virtex-II Pro FPGA eine maximale Taktfrequenz von 166,3 MHz.

Eine Performace-Analyse auf dem ChipIt Gold Edition Pro ergab mit einem Testbild eine Verarbeitungsrate von 574 PAL-Bilder pro Sekunde bei einem Systemtakt von 140 MHz. Wird das Modul hingegen nicht in einer Verfahrenskette eingesetzt, reduzierte sich die Bildrate auf 44 Bilder pro Sekunde da das Data-Interface zum Host-PC einen Flaschenhals darstellt.

