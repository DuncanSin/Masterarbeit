\chapter{Zusammenfassung und Ausblick}
\label{ch:sum}
\rm

\section{Zusammenfassung}\label{sec:summary}
Am Institut f�r Mikroelektronische Systeme der Leibniz Universit�t Hannover werden in einem Teilbereich die die Umsetzungsm�glichkeiten einer Musikklassifikation auf mobilen Endger�ten untersucht. Hierf�r werden Explorationen auf verschiedenen Prozessorachitekturen durch gef�hrt, mit dem Ziel einen Entwurfsraum zu definieren.\\\\
Im Rahmen dieser Arbeit wurden drei verschiedene Architekturen aus diesem Entwurfsraum auf ihre Einsetzbarkeit zur Musikklassifikation untersucht. Diese drei Architekturen wurden durch einen ARM Cortex A8, einen C674x DSP und einer heterogenen Architektur aus der Kombination aus beiden repr�sentiert. Diese sollten optimiert und auf Basis ihrer Laufzeit und Energieeffizenz bewertet werden.\\
Untersuchungen sollten hierf�r auf Basis von vier vorgegebenen Musikklassifikationsmethoden durchgef�hrt werden. Es stellte sich heraus, dass der zeitkritischste Schritt der Musikklassifikation in der Merkmalsextraktion bestand, daher wurden Optimierungen auch nur in diese Richtung durchgef�hrt. Nach der Optimierung wurden die Extraktionen auf dem Cortex A8 um Faktoren zwischen 2,5 (MCL3) und 22,5 (MCL4) je nach Methode beschleunigt. Auf dem C674x DSP wurden Beschleunigungen um Faktoren zwischen 4,2 (MCL3) und 10,5 (MCL4) erreicht. Hierbei muss erw�hnt werden, dass Extraktionen auf dem DSP im Allgemeinen schneller ausgef�hrt wurden, so dass die Extraktion von zum Beispiel MCL1 in den optimiertesten Versionen um einen Faktor von 1,8 schneller durchgef�hrt wird. Es scheint also, dass der DSP effektiver ist. Hierbei darf aber nicht au�er acht gelassen werden, dass der DSP nicht arbeiten kann, ohne dass der ARM zumindest im Idle-Modus ist. Somit hat der DSP einen h�heren Energieverbrauch und ist damit wieder ineffizienter. Eine Variante in der Prozessierung und Klassifikation ebenfalls auf dem DSP ausgef�hrt werden, wurde nicht untersucht.\\
F�r eine Kombination aus Cortex A8 und C674x wurde nur der sequenzielle Fall untersucht, so dass der DSP die Extraktion und danach der Cortex A8 Prozessierung und Klassifikation ausf�hrt. Auf Basis dieser Untersuchung wurde festgestellt, dass diese heterogene Architektur aus ARM und DSP ineffizienter ist, als wenn alle drei Schritte auf dem ARM ausgef�hrt werden, auch wenn der heterogene Ansatz schneller w�re.\\
Abschlie�end ist das Ergebnis dieser Arbeit, dass alle drei Architekturen f�r den Einsatz in einem mobilen Endger�t geeignet w�ren. Hierbei stellte sich der ARM als am energieeffizientesten heraus.

\section{Ausblick}\label{sec:vista}

Innerhalb dieser Arbeit wurden einige Aspekte der drei betrachteten Architekturen nicht betrachtet. Weiteres Forschungen sollten daher in Richtung des parallelen Einsatzes von ARM und DSP gehen. Hierbei sollte eine Verwendung des DSP zur Merkmalsextraktion angestrebt werden, bei der der DSP mehrere Musikst�cke verarbeitet und danach der ARM diese Prozessiert und Klassifiziert. In dieser Zeit sollte der DSP weitere Lieder verarbeiten.\\
Des weiteren wurden auch nicht alle Optimierungsm�glichkeiten in vollem Ausma� ausgesch�pft, so dass auch in diese Richtung weitere Forschung geschehen sollte.