\chapter{Zusammenfassung und Ausblick}
\label{ch:sum}
\rm

\section{Zusammenfassung}\label{sec:summary}
Am Institut f�r Mikroelektronische System der Leibniz Universit�t Hannover werden in einem Projekt Algorithmen und Architekturen zur Konzeption Energie-effizienter Systeme zur Audio-Signalklassifikation untersucht. Hierbei wird der Entwurfsraum von Hardware-Architekturen zur Audio-Signalklassifikation insbesondere f�r mobile Endger�te exploriert.

Im Rahmen dieser Arbeit wurde eine heterogene Prozessorarchitektur der Firma Texas Instruments hinsichtlich der Laufzeiten und der Energie-Effizienz bei Anwendung der inhaltsbasierten Musikklassifikation untersucht. Hierbei wurde die rechenintensive Merkmalsextraktion von vier verschiedenen Musikklassififkationsmethoden f�r den ARM- und den DSP-Prozessor optimiert.

Die Optimierungsma�nahmen beschleunigen die Extraktionen auf dem Cortex A8 um Faktoren zwischen 2,5 (MCL3) und 22,5 (MCL4). Die Extraktion auf dem C674x DSP wird mit Faktoren zwischen 4,2 (MCL3) und 10,5 (MCL4) schneller durchgef�hrt als beim implementierten Referenz-Programm. Der C674x DSP ist somit besser f�r die Extraktion von Merkmalen geeignet als der ARM Cortex A8.

F�r die Analyse des heterogenen Systems wurde die Ausf�hrung der Musikklassifikation dahingehend realisiert, dass Merkmalsextraktion auf dem C674x DSP und Prozessierung und Klassifikation auf dem ARM Cortex A8 ausgef�hrt wurden. Die Ausf�hrung der drei Schritte geschah dabei sequentiell.

Eine Musikklassifikation wird auf dem heterogenen System um das 2- (MCL3) bis 8-fache (MCL1) schneller verarbeitet als auf dem Cortex A8. Daraus folgt, dass unter dem Aspekt der Verarbeitungszeit das heterogene System vorzuziehen ist.\\
Im Bezug auf die Energie-Effizienz ist der ARM Cortex A8 von 13\% bei MCL2 bis 84\% effizienter als das heterogene System. Die einzige Ausnahme bildet MCL3, bei der das heterogene System um 42\% effizienter ist.\\


%\section{Ausblick}\label{sec:vista}
%
%Innerhalb dieser Arbeit wurde eine parallele Nutzung der beiden Prozessorkerne nicht betrachtet. Weiteres Forschungen sollten daher in Richtung des parallelen Einsatzes von ARM und DSP gehen. Hierbei sollte eine Verwendung des DSP zur Merkmalsextraktion angestrebt werden, bei der der DSP mehrere Musikst�cke verarbeitet und danach der ARM eine Prozessierung und Klassifikation dieser durchf�hrt, w�hrend weitere Musikst�cke auf dem DSP extrahiert werden.\\
%Des weiteren wurden nicht alle Optimierungsm�glichkeiten in vollem Ausma� ausgesch�pft, so dass auch in diese Richtung weitere Forschung geschehen sollte. Ans�tze f�r weiteres Optimierungspotenzial der Merkmalsextraktion auf ARM Cortex A8 und TI C674x DSP wurden in der Evaluation herausgestellt.