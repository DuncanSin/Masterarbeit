\chapter{Zusammenfassung und Ausblick}
\label{ch:sum}
\rm

\section{Zusammenfassung}\label{sec:summary}
Am Institut f�r Mikroelektronische System der Leibniz Universit�t Hannover werden in einem Projekt Algorithmen und Architekturen zur Konzeption Energie-effizienter Systeme zur Audio-Signalklassifikation untersucht. Hierbei wird der Entwurfsraum von Hardware-Architekturen zur Audio-Signalklassifikation insbedondere f�r mobile Endger�te exploriert.

Im Rahmen dieser Arbeit wurde eine heterogene Prozessorarchitektur der Firma Texas Instruments hinsichtlich der Laufzeiten und der Energie-Effizienz bei Anwendung der inhaltsbasierten Musikklassifikation untersucht. Hierbei wurde die rechenintensive Merkmalsextraktion von vier verschiedenen Musikklassififkationsmethoden f�r den ARM- und den DSP-Kern optimiert.

Nach der Optimierung wurden die Extraktionen auf dem Cortex A8 um Faktoren zwischen 2,5 (MCL3) und 22,5 (MCL4) beschleunigt. Die Beschleunigung der Extraktion auf dem C674x DSP geschah mit Faktoren zwischen 4,2 (MCL3) und 10,5 (MCL4)schneller als das portierte Referenz-Programm. Allgemeinen wird die Merkmalsextraktion auf dem DSP schneller ausgef�hrt als auf dem ARM Cortex A8.\\
Eine Musikklassifikation wird auf dem heterogenen System um das 1,2- bis 3,1-fache schneller verarbeitet als auf dem Cortex A8. Daraus folgt, dass unter dem Aspekt der Verarbeitungszeit das heterogene System das effektivere ist.\\
Wird die Energie-Effizienz betrachtet, so wurde festgestellt, dass der ARM Cortex A8 um das ... effizienter ist als heterogene System.\\

%\section{Ausblick}\label{sec:vista}
%
%Innerhalb dieser Arbeit wurde eine parallele Nutzung der beiden Prozessorkerne nicht betrachtet. Weiteres Forschungen sollten daher in Richtung des parallelen Einsatzes von ARM und DSP gehen. Hierbei sollte eine Verwendung des DSP zur Merkmalsextraktion angestrebt werden, bei der der DSP mehrere Musikst�cke verarbeitet und danach der ARM eine Prozessierung und Klassifikation dieser durchf�hrt, w�hrend weitere Musikst�cke auf dem DSP extrahiert werden.\\
%Des weiteren wurden nicht alle Optimierungsm�glichkeiten in vollem Ausma� ausgesch�pft, so dass auch in diese Richtung weitere Forschung geschehen sollte. Ans�tze f�r weiteres Optimierungspotenzial der Merkmalsextraktion auf ARM Cortex A8 und TI C674x DSP wurden in der Evaluation herausgestellt.