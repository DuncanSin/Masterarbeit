\chapter{Zusammenfassung und Ausblick}
\label{ch:sum}
\rm

\section{Zusammenfassung}\label{sec:summary}

In dieser Arbeit ging es darum Algorithmen zur Klassifizierung von Musikst�cken auf einer heterogenen Prozessorarchitektur zu implementieren und zu optimieren. Hierf�r wurde zu erst die hierf�r gew�hlte Plattform in \textbf{Kapitel \ref{ch:board}} vorgestellt, ein EVM8168-Evaluationsboard der Firma \textit{Texas Instruments} mit einem DaVinci\texttrademark-Prozessor (DM8168), der aus einem ARM-Prozessor und einem DSP besteht. \\
Danach wurden in \textbf{Kapitel \ref{ch:optarm}} zun�chst die Bottlenecks einer rein ARM-seitigen Implementierung extrahiert und deren Optimierungen diskutiert. Hierbei wurde die Fouriertransformation, die zur Berechnung der Features des Frequenzbereichs ben�tigt wird, als gr��ster Bottleneck identifieziert und versucht diese durch einbinden der NEON-Einheit, die der Prozessor zur verf�gung stellt, zu optimieren.

\section{Ausblick}\label{sec:vista}