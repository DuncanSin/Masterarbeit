\chapter{Zusammenfassung und Ausblick}
\label{ch:sum}
\rm

\section{Zusammenfassung}\label{sec:summary}
Am Institut f�r Mikroelektronische Systeme der Leibniz Universit�t Hannover werden in einem Teilbereich die die Umsetzungsm�glichkeiten einer Musikklassifikation auf mobilen Endger�ten untersucht. Hierf�r werden Explorationen auf verschiedenen Prozessorachitekturen durch gef�hrt, mit dem Ziel einen Entwurfsraum energieeffizenter Architekturen zu definieren.\\\\
Im Rahmen dieser Arbeit wurden die heterogene Prozessorarchitektur des DaVinci Video Prozessors der Firma Texas Instruments aus diesem Entwurfsraum auf ihre Einsetzbarkeit zur Musikklassifikation untersucht. Diese Architektur besteht aus einem ARM Cortex A8 und einem TI C674x DSP. Die Merkmalsextraktion sollte f�r diese beiden Prozessoren implementiert und optimiert werden. Anschlie�end sollte eine Untersuchung der Energieeffizienz f�r den gemeinsamen Einsatz dieser beiden Prozessoren durchgef�hrt werden.\\
Untersuchungen sollten hierf�r auf Basis von vier vorgegebenen Musikklassifikationsmethoden durchgef�hrt werden. Es stellte sich heraus, dass der zeitkritischste Schritt der Musikklassifikation wie erwartet in der Merkmalsextraktion bestand, daher wurden Optimierungen in diese Richtung vorgenommen. Nach der Optimierung wurden die Extraktionen auf dem Cortex A8 um Faktoren zwischen 2,5 (MCL3) und 22,5 (MCL4) je nach Methode beschleunigt. Auf dem C674x DSP wurden Beschleunigungen um Faktoren zwischen 4,2 (MCL3) und 10,5 (MCL4) erreicht. Hierbei muss erw�hnt werden, dass Extraktionen auf dem DSP im Allgemeinen schneller ausgef�hrt wurden, so dass die Extraktion von zum Beispiel MCL1 in den optimiertesten Versionen um einen Faktor von 1,8 schneller durchgef�hrt wird. Zur reinen Extraktionszeit des DSP wird auch noch eine Zeit ben�tigt, die durch die Kommunikation zwischen ARM und DSP entsteht. F�r diese Kommunikation wurde eine Zeit von 4 ms ermittelt.\\
F�r eine Kombination aus Cortex A8 und C674x wurde nur der sequenzielle Fall untersucht, so dass der DSP die Extraktion und danach der Cortex A8 Prozessierung und Klassifikation ausf�hrt. Auf Basis dieser Untersuchung wurde festgestellt, dass diese heterogene Architektur aus ARM und DSP ineffizienter ist, als eine alleinige Nutzung des Cortex A8 f�r alle drei Schritte. Auch wenn der heterogene Ansatz eine geringere Laufzeit aufweist, so ist doch die Energieeffizienz durch den zus�tzlichen Energieverbrauch des DSPs geringer.\\
Das Ergebnis dieser Arbeit ist somit, dass eine Umsetzung der Musikklassifikation auf der Architektur des ARM Cortex A8 energieeffizienter ist, als eine Umsetzung auf der heterogenen Architektur bestehend aus ARM Cortex A8 und TI C674x DSP.

\section{Ausblick}\label{sec:vista}

Innerhalb dieser Arbeit wurde eine parallele Nutzung der beiden Prozessorkerne nicht betrachtet. Weiteres Forschungen sollten daher in Richtung des parallelen Einsatzes von ARM und DSP gehen. Hierbei sollte eine Verwendung des DSP zur Merkmalsextraktion angestrebt werden, bei der der DSP mehrere Musikst�cke verarbeitet und danach der ARM eine Prozessierung und Klassifikation dieser durchf�hrt, w�hrend weitere Musikst�cke auf dem DSP extrahiert werden.\\
Des weiteren wurden nicht alle Optimierungsm�glichkeiten in vollem Ausma� ausgesch�pft, so dass auch in diese Richtung weitere Forschung geschehen sollte. Ans�tze f�r weiteres Optimierungspotenzial der Merkmalsextraktion auf ARM Cortex A8 und TI C674x DSP wurden in der Evaluation herausgestellt.