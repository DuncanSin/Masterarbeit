\documentclass[12pt,a4paper,oneside,titlepage]{report}

%\textwidth=15cm
%\hoffset=-1 cm
\usepackage{ngerman}
\usepackage[latin1]{inputenc}
\usepackage{ae}
\usepackage{graphicx}
\usepackage{amsmath}
\usepackage{amsthm}
\usepackage{dsfont}
\usepackage{listings}
\lstset{language=C}
%\usepackage{amsfonts,textcomp}
%\usepackage{calc}

\usepackage{makeidx,showidx} \makeindex
\usepackage{multirow}
\usepackage{tabularx}
\usepackage{subfigure}
\usepackage{epic}
\usepackage{longtable}
\usepackage{rotating}
\usepackage{float}
\usepackage{fancyvrb}
%\usepackage{glossary}
%URL Farb und Link Anpassung
%\usepackage{fancyhdr}
\usepackage{color}
\usepackage{hyperref}
\usepackage{lscape}
\definecolor{black}{rgb}{0,0,0}
\definecolor{darkblue}{rgb}{0,0,0.5}
\hypersetup{colorlinks, linkcolor=black, urlcolor=darkblue, citecolor=black, pdftitle={Implementierung und Optimierung von Algorithmen zur Audiosignal-Klassifikation auf einer heterogenen Prozessorarchitektur}, pdfauthor={Kristian Wolpers}, pdfsubject={Masterarbeit, Juli 2013}}


\renewcommand{\arraystretch}{1.5}
\newcommand\textsubscript[1]{\ensuremath{{}_{\text{#1}}}}

\addtolength{\headheight}{0,3cm}
\addtolength{\headsep}{-0,3cm}
\setlength{\parindent}{0pt}
\setlength{\parskip}{12pt plus 6pt minus 3pt}
\setlength{\partopsep}{0pt}
\addtolength{\topmargin}{-0,6cm}
\addtolength{\oddsidemargin}{-0,6cm}
\addtolength{\textwidth}{2cm}
\addtolength{\textheight}{2cm}

\restylefloat{figure}
\IfFileExists{url.sty}{\usepackage{url}}

\usepackage{fancyhdr}
\pagestyle{fancy}


\renewcommand{\headrulewidth}{0.4pt}
\renewcommand{\sectionmark}[1]{\markright{ Kapitel  \thesection\ \it #1}}
\renewcommand{\chaptermark}[1]{\markright{ Kapitel  \thechapter\ \it #1}}

\lhead{\nouppercase{\rightmark}}
\rhead{\thepage}
\cfoot{}

% Damit auch die Titelseiten fancy sind!
\makeatletter
\let\ps@plain=\ps@empty
\makeatother

%Fourier-Korrespondenz link  *-o

\newcommand{\ftranl}{\bullet\!\!-\!\!\circ}

%

%Fourier-Korrespondenz rechts  o-*

\newcommand{\ftranr}{\circ\!\!-\!\!\bullet} 

%

%Fourier-Korrespondenz unten o

%                            |

%                            *

\newcommand{\ftranb}{{\setlength{\unitlength}{12pt}\begin{picture}(0.5,1.6)\put(0,1){$\circ$}\put(0.186,0.186){\rule[2.5pt]{0.6pt}{7.5pt}}\put(0,0){$\bullet$}\end{picture}}}

%

%Fourier-Korrespondenz oben  *

%                            |

%                            o

\newcommand{\ftrant}{{\setlength{\unitlength}{12pt}\begin{picture}(0.5,1.6)\put(0,1){$\bullet$}\put(0.186,0.186){\rule[2.5pt]{0.6pt}{7.5pt}}\put(0,0){$\circ$}\end{picture}}}

\begin{document}

% Listing Package definieren
\lstset{
framexleftmargin=6mm,
framexrightmargin=0mm,
framextopmargin=2mm,
framexbottommargin=2mm,
frame=lines,
basicstyle=\scriptsize, % \footnotesize,
keywordstyle=\bfseries,
identifierstyle=,
commentstyle=\color{white},
%stringstyle=\ttfamily,
showstringspaces=false,
numbers=left,
numberstyle=\tiny,
xleftmargin=18pt
%morekeywords={MV,MACI_U8,MACPLZI_U8,SUB_16,PERMREG0_8,ABSADDI_8,CLIP_U8,SMVI,SUBICS_8,SUBCR_8,MVCR_8,V0R10,V1R23,V1R7,V0R28,V1R4,V1R0,V1R1,V0CONDSEL}
}

\pagenumbering{alph}

\thispagestyle{empty} 
 \begin{large}
  \begin{center}
  %\includegraphics*[scale=0.3]{bilder/TI_neubau}\\[0.5 cm]
   \Large \textbf{Leibniz Universit�t Hannover}\\
   \Large \textbf{Institut f�r Mikroelektronische Systeme}\\
   \Large \textbf{Prof. Dr.-Ing. H. Blume}\\[4.5 cm]
%  \LARGE Studienarbeit\\[0.8 cm]
%   \Large \textsc{Leibniz Universit�t Hannover}\\
%   \Large \textsc{Institut f�r Mikroelektronische Systeme}\\
%   \Large \textsc{Prof. Dr.-Ing. P. Pirsch}\\[4.5 cm]

   \LARGE \textbf{Konzeption und Evaluation von Instruktionssatzerweiterungen zur Optical-Flow-Berechnung f�r einen ASIP}\\
    [5.5 cm]

   \Large \textbf{Bachelorarbeit}\\
   \large \textbf{von}\\
   \Large \textbf{Kristian Wolpers}\\[2.8 cm]



   \Large \textbf{Oktober 2010}\\
  \end{center}
 \end{large}

\input{Misc/clearpage}
\newpage 
\thispagestyle{empty}
 \begin{large}
  \begin{center}
%  \includegraphics*[scale=0.3]{abbildungen/unilogo}\\[0.5 cm]
   \Large \textbf{Leibniz Universit�t Hannover}\\
   \Large \textbf{Institut f�r Mikroelektronische Systeme}\\
   \Large \textbf{Prof. Dr.-Ing. H. Blume}\\[4.5 cm]
%   \Large \textsc{Leibniz Universit�t Hannover}\\
%   \Large \textsc{Institut f�r Mikroelektronische Systeme}\\
%   \Large \textsc{Prof. Dr.-Ing. P. Pirsch}\\[4.5 cm]
%  \LARGE Studienarbeit\\[0.8 cm]
   \LARGE \textbf{Platzhalter}\\
    [5.5 cm]

   \Large \textbf{Masterarbeit}\\
   \large \textbf{von}\\
   \Large \textbf{Kristian Wolpers}\\[2.8 cm]

   \Large \textbf{Betreuer: Dipl.-Ing. I. Schm�decke}\\
   \Large \textbf{Erstpr�fer: Prof. Dr.-Ing. H. Blume}\\
   \Large \textbf{Zweitpr�fer: Prof. Dr.-Ing. C. M�ller-Schloer}\\

  \end{center}
 \end{large}

















\thispagestyle{empty}

\vspace*{18cm}

\par
Ich versichere, dass ich die vorgelegte Arbeit selbstst�ndig verfasst und keine anderen als die angegebenen Quellen, Hilfen und Hilfsmittel benutzt habe. 
%\\[\baselineskip]

\vspace*{1cm}

\par
Hannover, den 28. Oktober 2010 %\today
%\include{Verschwiegenheit}
\pagenumbering{Roman}                   % R�mische Seitennummern �ber den Vorspann
\setcounter{secnumdepth}{5}             % Nur section und subsection numerieren
%Inhaltsverzeichnis erstellen                                        
{\setlength{\parskip}{2pt}
\pdfbookmark[1]{Inhaltsverzeichnis}{toc}
\setcounter{tocdepth}{3}
\tableofcontents}                   
%Abbildungsverzeichnis erstellen
{\setlength{\parskip}{2pt}
\listoffigures}
\newpage
%Tabellenverzeichnis erstellen
{\setlength{\parskip}{2pt}
\listoftables}
\newpage
\pagenumbering{arabic}

\sloppy                            % be a bit more tolerant
\hbadness2000 \vbadness2000

% Disable single lines at the start of a paragraph (Schusterjungen)
\clubpenalty = 10000
%
% Disable single lines at the end of a paragraph (Hurenkinder)
\widowpenalty = 10000 \displaywidowpenalty = 10000

\chapter{Einf�hrung}
\label{ch:intro}
\rm

\chapter{Grundlagen}
\label{ch:grund}
\rm

\section{Algorithmen}\label{sec:alg}
In diesem Kapitel sollen die Extraktionsalgorithmen vorgestellt werden, aus denen die im \textbf{Kapitel \ref{ch:optimization}} beschriebenen FeatureSets zusammengesetzt sind, die f�r die Laufzeitanalyse des Codes verwendet werden. Hierf�r werden diese in zwei Bereiche unterteilt, einmal die Features, welche im Zeitbereich arbeiten (\textbf{\ref{subsec:zeit}}) und diese die im Frequenzbereich arbeiten (\textbf{\ref{subsec:freq}}). Diese Unterteilung wurde aus \cite{haller} �bernommen.


\subsection{Features des Zeitbreichs}\label{subsec:zeit}

\subsubsection{LowEnergy}\label{subsubsec:low}

\subsubsection{RootMeanSquare}\label{subsec:root}

\subsubsection{ZeroCrossingRate}\label{subsubsec:zero}


\subsection{Features des Frequenzbereichs}\label{subsec:freq}

\subsubsection{FFT}\label{subsec:fft}

\subsubsection{Chromavektor}\label{subsubsec:Chromavektor}

\subsubsection{AmplitudeOfMaximumInChromagram}\label{subsec:amp}

\subsubsection{Magnitude}\label{subsec:mag}

\subsubsection{MFCC}\label{subsec:mfcc}

\subsubsection{NormalizedAudioSpectrumEnvelope}\label{subsec:nase}

\subsubsection{OctaveSpectralContrast}\label{subsec:osc}

\subsubsection{SpectralCentroid}\label{subsec:speccen}

\subsubsection{SpectralCrestFactor}\label{subsec:speccr}

\subsubsection{SpectralFlux}\label{subsec:specfl}

\subsubsection{SpectralRolloff}\label{subsec:specrol}

\subsubsection{SubBandEnergyRatio}\label{subsec:sub}



\section{Profiling}\label{sec:prof}
Der Begriff Profiling bescheibt Methoden, mit denen das Verhalten von Applikationen auf Systemebene analysiert werden k�nnen, hierzu z�hlen Analysen der Laufzeit, der Schreib- und Lesezugriffe und auch die Verh�ltnisse von Instuktionen und Takten zueinander. Im Folgenden werden nun Beispielhaft die Profilingmethoden der Laufzeitmessung, der richtigen Vorhersagen des Vorladens in den Cache (sog. Cache Hits), sowie die Betrachtungen des Verh�ltnisses von Takt und Instuktionen (Instructions per Cycle (IPC) und Cycle per Instruction (CPI)).

\subsection{Laufzeitanalyse}\label{subsec:time}

\subsection{Cache Hits}\label{subsec:cache}

\subsection{IPC und CPI}\label{subsec:ipccpi}


\chapter{Das EVM8168-Entwicklungsboard}
\label{ch:board}
\rm

F�r die in den folgenden Kapiteln beschriebenen Arbeitsschritte zur Optimierung und Analyse des Programmes wurde ein EVM8168-Entwicklungsboard verwendet, welches von der Firma Texas Instruments in Zusammenarbeit mit der Firma Spectrum Digital entwickelt wurde.
Dieses Board kann mit Hilfe eines DM816x (DaVinci\texttrademark) ARM-Prozessors entweder selber Programme ausf�hren oder es k�nnen auch die beiden ARM-Prozessoren C6A816x (Integra\texttrademark) oder AM389x (Sitara\texttrademark) emuliert werden. 


\section{Aufbau des EVM8168} \label{sec:evm816}
Wie in \cite{spec} beschrieben bietet das EVM816x-Entwicklungsboard eine Standalone-Plattform um Programme f�r DaVinci\texttrademark, Integra\texttrademark~oder Sitara\texttrademark~Prozessoren der Firma Texas Instruments zu entwickeln und zu debuggen. Hierf�r sind neben dem DaVinci\texttrademark~noch weitere On-Board Peripherie auf dem Board aufgebracht, die im folgenden teilweise n�her erkl�rt werden sollen.
Das EVM8168-Board hat unter anderem folgende Komponenten integriert:

\begin{itemize}
	\item DM816x- (DaVinci\texttrademark-)ARMprozessor (\textbf{Kapitel~\ref{sec:davinci}}) mit NEON-Einheit (\textbf{Kapitel~\ref{subsec:neon}}) und DSP (\textbf{Kapitel~\ref{subsec:dsp}})
	\item 1 GB DDR3-RAM
	\item AC31061-Audiochip
	\item Gigabit Ethernet
	\item HDMI
	\item VGA
	\item USB

\end{itemize}

\textbf{Abbildung~\ref{fig:top_ti816x_evm}} zeigt eine Draufsicht auf das Entwicklungsboard und die unterhalb dessen angebrachte Daughtercard mit weiteren Anschlussm�glichkeiten.

\begin{figure}[htbp]
	\centering
		\includegraphics[scale=0.4]{../Pictures/top_ti816x_evm.png}
	\caption{Draufsicht auf das EVM8168}
	\label{fig:top_ti816x_evm}
\end{figure}



\section{Der DaVinci\texttrademark}\label{sec:davinci}
Bei dem auf dem EVM816x verwendeten ARM-Prozessor DM816x handelt es sich um einen eigentlich f�r die Videoprozessirrung optimierten Prozessor der DaVinci\texttrademark-Familie von Texas Instruments.\\
Der DM816x ist ein Dualcore-Prozessor, der aus einer heterogenen Architektur basiert. \\
Zum einen enth�lt dieser einen ARM\textregistered Cortex\texttrademark-A8 Risc Prozessor mit bis zu 1,35 GHz Taktung, welcher auf der ARMv7 Architektur basiert, was bedeutet, dass es sich um einen superscalaren Prozessorkern handelt und er der NEON\texttrademark Multimedia Architektur entspricht. Dieser besitzt 32K-Byte Instruction- und Datacaches, sowie einen 256K-Byte gro�en L2 Cache. Des weiteren sind noch 64K-RAM und 48K-Byte Boot ROM vorhanden.\\
Als zweiter Prozessorkern enth�lt der DM816x einen C674x VLIW DSP mit bis zu 1,125 GHz Taktung. Das besondere An diesem DSP ist, dass der vollst�ndig softwarekompatibel mit DSPs der C67x+\texttrademark und C64x+\texttrademark Reihen von Texas Instruments ist (n�heres in \textbf{\ref{subsec:dsp}})\cite{evm8168}.

\subsection{Die NEON-Einheit}\label{subsec:neon}
\subsection{Der C674x-DSP-Prozessor}\label{subsec:dsp}

Der C674x ist ein Floating-Point VLIW DSP mit 64 General-Purpose Registern mit je 32-Bit. Er besitzt sechs ALU Funktionseinheiten mit 32 und 40 Bit. Er unterst�tzt 32-Bit Floating Point Integer mit IEEE Single Precision (SP 32-Bit) und IEEE Double Precision (DP 64-Bit). Auf diesen schafft er bis zu vier SP Adds pro Takt und vier DP Adds alle zwei Takte, des weiteren kann er bis zu zwei Floating-Point approximierte reziproke oder quadratische Wurzeln pro Takt in SP oder DP berechnen.\\
Au�erdem besitzt er zwei Multiplizierer, entweder gemischt pr�zise Floating-Point oder Fixed-Point Integer berechnen k�nnen. Im Floating-Point-Modus schaffen diese folgende Berechnungen in den angegebenen Takten:

\begin{itemize}
	\item $2~SP \times SP~\rightarrow~ SP~pro~Takt$
	\item $2~SP \times SP~\rightarrow~ DP~pro~zwei~Takte$
	\item $2~SP \times DP~\rightarrow~ DP~pro~drei~Takte$
	\item $2~DP \times DP~\rightarrow~ DP~pro~vier~Takte$
\end{itemize}

Im Fixed-Point-Modus sind zwei 32x32, vier 16x16 oder acht 8x8 Multiplizierungen pro Takt m�glich.\\
Die Speicherarchitektur des C674x besteht aus zwei Ebenen, auf der Ersten sind 32K-Byte L1P und L1D RAM und Cache, auf der Zweiten 256K-Byte L2 RAM und Caches.\\
In der hier vorliegenden Architektur fungiert der DSP als Slave-Prozessor \cite{evm8168}.\\
Das Blockdiagramm des C674x kann \textbf{Abbildung \ref{fig:bddsp}} entnommen werden.

\begin{figure}[htbp]
	\centering
		\includegraphics[width=1.00\textwidth]{../Pictures/DSPBlock.pdf}
	\caption{Blockdiagramm des C764x\cite{evm8168}}
	\label{fig:bddsp}
\end{figure}

\chapter{Implementierung \& Optimierung}
\label{ch:opt}
\rm

In diesem Kapitel werden die Implementierung und die Optimierung des Referenz-Programm-Codes auf dem ARM Cortex A8 und dem C674x DSPs beschrieben.\\
Hierf�r wir in \textbf{Abschnitt \ref{sec:str}} als erstes die Optimierungsstrategie vorgestellt, nach der die Optimierungen in den folgenden Abschnitten durchgef�hrt werden. Anschlie�end werden in \textbf{Abschnitt \ref{sec:optarm}} die Implementierung und Optimierung des Referenz-Programm-Codes auf dem ARM Cortex A8 und in \textbf{Abschnitt \ref{sec:optdsp}} die auf dem C674x DSP vorgestellt.

\section{Optimierungsstrategie}\label{sec:str}
Das Ziel der in den folgenden Abschnitten vorgestellten Optimierungen ist die Verbesserung der Laufzeit der Musikklassifikation. Hierf�r wird angestrebt, die von den Prozessoren zur Verf�gung gestellten Rechenressourcen effizient auszunutzen.\\
F�r die Berechnung dieser Effizienz sind mehrere G�tema�e vorhanden, unter anderem Werte f�r \textit{Instruktionen pro Takt} (engl. Instruction per Cycle oder kurz IPC), \textit{Takte pro Instruktion} (engl. Cycles per Instruktion oder kurz CPI), die Messung von \textit{Cache Hits und Misses} und die \textit{Laufzeit} des Programms. Aufgrund der von Texas Instrument zur Verf�gung gestellten Werkzeuge ist eine Messung von IPC/CPI-Werten oder Cache Hits/Misses nicht ohne weiteres m�glich, so dass in dieser Arbeit auf die Messung der Laufzeit zur Bewertung der Effizienz zur�ckgegriffen wird.\\
Um bei der Optimierung den gr��ten Gewinn zu erreichen, werden gezielt die Programmteile optimiert, die den gr��ten Anteil an der Verarbeitungszeit des Musikklassifikators besitzen. Hierf�r wird als erstes aus den drei Schritten der Musikklassifikation derjenige identifiziert, der am Laufzeit-intensivsten ist. Als n�chstes werden aus diesem Schritt ebenfalls jene Algorithmen identifiziert und optimiert, die den gr��ten Anteil an der Verarbeitungszeit dieses Schrittes einnehmen. Auf diese Weise wird versucht, mit gezielten Optimierungen gro�e Verbesserungen an der Gesamtverarbeitungszeit zu erreichen. 
Als n�chstes muss entschieden werden, wie Optimierungspotenziale identifiziert werden.\\
Sowohl die Untersuchungen auf dem ARM Cortex-A8 als auch die auf dem C674x DSP werden hierf�r in drei Schritten durchgef�hrt:

\begin{enumerate}
\item Laufzeitanalyse des Programmteils zur Identifikation zeitkritischer Algorithmen oder Codesegmente
\item Optimierung der gefundenen Algorithmen oder Codesegmente
\item Bewertung der durchgef�hrten Optimierungen
\end{enumerate}

Hierbei handelt es sich um einen iterativen Prozess. Das hei�t, dass nach einer durchgef�hrten Optimierung eine weitere Laufzeitmessung durchgef�hrt wird. Dieses geschieht zum einen f�r die Bewertung einer durchgef�hrten Optimierung und zum anderen, um eventuell neue zeitkritische Algorithmen zu identifizieren.


\section{Optimierung des ARM}

\subsection{Laufzeitmessung des Gesammtprogramms}

\subsection{Laufzeitmessung der Extraktion}

\subsection{Ansatzpunkte und Bottlenecks}\label{sec:ansatz}


\subsubsection{FFT}


\subsection{Libav als Optimierung der FFT}\label{subsec:optFFT}
\subsubsection{Laufzeitmessung}
\subsubsection{Aufbau der FFT}
\subsubsection{Einbindung}

\subsection{Optimierung der Amplitude of Spectrum}\label{subsec:optAOS}
\subsubsection{Laufzeitmessung}
\subsubsection{Codeanpassung}

\subsection{Optimierung von MFCC}

\subsection{Optimierung der Zero Crossing Rate}



 
\chapter{Optimierung des DSP-Codes}
\label{ch:optdsp}
\rm

In diesem Kapitel soll die Planung und Optimierung des DSP-Codes beschreiben werden.\\
Wie bereits in \textbf{Kapitel \ref{sec:ansatz}} gezeigt wurde, nimmt die Extraktion der zu untersuchenden Features einen Gro�teil der Laufzeit des Programmes ein.
Unter diesem Gesichtspunkt und der in \textbf{Kapitel \ref{sec:davinci}} gezeigten zugrundeliegenden heterogenen Architektur des zu betrachtenden Systems, liegt es nahe, die Extraktion auf dem DSP auszuf�hren.\\
In den nun folgenden Abschnitten sollen daher, wie schon im vorhergehenden Kapitel erst die Bottlenecks identifiziert (\textbf{\ref{sec:ansatzdsp}}) und dannach die durchgef�hrten Optimierungen beschrieben werden (\textbf{\ref{sec:mathlib}} - \textbf{\ref{sec:compiler}}).

\section{Bottlenecks des DSP-Codes}\label{sec:ansatzdsp}

F�r die Laufzeitmessungen des DSP-Codes wurden wie schon im vorherigen Kapitel die FeaturesSets 1-4 (\textbf{\ref{subsec:fset1}} - \textbf{\ref{subsec:fset4}}) verwendet.
Die ben�tigen Zeiten wurden auch hier wieder mit der Gesammtzeit in verh�ltnis gesetzt, wodurch die aus \textbf{Abbildung \ref{fig:dsp}} zu entnehmenden Anteile der einzelnen Features an der Gesammtlaufzeit errechnet wurden.

\begin{figure}
	\centering
		\includegraphics[width=1.00\textwidth]{../Pictures/ResultsExtraktionDSP.pdf}
	\caption{Laufzeitanteile der Features auf dem DSP}
	\label{fig:dsp}
\end{figure}

Es ist deutlich zu erkennen, dass in den FeatureSets 1 - 2d und 4 \textit{Amplitude of Spectrum} (\textbf{\ref{subsubsec:aos}}) und im FeatureSet 3 \textit{Root Mean Square} (\textbf{\ref{subsubsec:rms}}) die meiste Laufzeit in Anspruch nehmen, teilweise mit Anteilen weit �ber 50\%. Da beide Features eigentlich nur aus Summationen, Divisionen und der Bildung von Quadratwurzeln bestehen, die Bestandteil der Standartbibliotheken sind, scheint bei der Ausf�hrung von mathematischen Funktionen ein Bottleneck zu entstehen. Desweiteren ist zu sehen, dass auch bei der Umsetzung des Codes auf dem DSP ein weiterer Bottleneck bei der Ausf�hrung der FFT zu bestehen scheint, da auch diese mit ca. 30\% der Ausf�hrungszeit zu Buche schl�gt.

\section{Optimierung der Rechenfunktionen mit MATHLIB}\label{sec:mathlib}

\section{Optimierung der FFT mit DSPLIB}\label{sec:dsplib}

\section{Optimierung f�r den Compiler}\label{sec:compiler} 
\chapter{Evaluation}
\label{ch:results}
\rm

\section{Einf�hrung}

\begin{itemize}
\item Einf�hren der Bewertungsmetriken:
\begin{itemize}
\item Extraktionsrate: $\frac{Fenster}{Zeit}$
\item Effizienz: $\frac{Geschwindigkeit}{Ennergieverbrauch}$
\end{itemize}
\end{itemize}

\section{Evaluation der ARM-Optimierung}

\subsection{Qualit�t der Optimierung}

\begin{itemize}
\item Extraktionsraten der FeatureSets
\item Effizienz der FeatureSets
\end{itemize}

\subsection{Weitere Optimierungspotenziale}

\begin{itemize}
\item Sortierung (76\%) bei OSC
\item DCT (37,6\%) bei MFCC, Logarithmus (38\%) nicht viel machbar
\item Signum (90\%) bei ZCR nicht viel machbar
\item MAG von Hand optimieren (Intrinsics oder ASM)
\item CV nicht viel machbar, da Einordnungsteil zu unvorhersagbar
\item HW mit ASM, spart eventuell Lade- und Speicherbefehle (Durchg�ngiger Datenpfad)
\item andere Eventuell auch parallelisierbar mit NEOn, aber irrelevant, weil zu geringe Laufzeiten
\end{itemize}

\section{Evaluation der DSP-Optimierung}

\subsection{Qualit�t der Optimierung}
\begin{itemize}
\item Extraktionsraten der FeatureSets
\item Effizienz der FeatureSets
\end{itemize}

\subsection{Weitere Optimierungspotenziale}

\begin{itemize}
\item Sortierung (55,5\%) bei OSC
\item Akkumulation (73,6\%) mit ASM und DCT (23,6\%) bei MFCC
\end{itemize}

\section{Evaluation heterogenes System}

\subsection{Einschr�nkungen}

\begin{itemize}
\item Taktzahlgrenzen der Prozessoren => H�herer Takt = H�herer Energieverbrauch => evtl. ineffizienter
\item Kommunikation, bei zu gro�en Lieder auch iteratives Schreiben in den Shared-Speicher => Kostet Zeit
\end{itemize}

\subsection{Gegen�berstellung ARM und ARM+DSP}

\begin{itemize}
\item Laufzeitdiskusion
\item Leistungsverbrauch
\item Effizienz
\end{itemize}

\section{Vorstellung ARM als heterogenes System}

$\;$ \\
$\;$ \\
$\;$ \\
$\;$ \\
$\;$ \\
$\;$ \\
$\;$ \\
$\;$ \\
$\;$ \\
$\;$ \\

\section{Fazit}

\begin{itemize}
\item ARM+DSP zu ineffizient
\begin{itemize}
\item Kommunikationszeit
\item Erh�hter Energiebedarf durch Zuschalten von DSP
\item DSP bringt keinen wirklichen Geschwindigkeitszuwachs
\end{itemize}
\item ARM+NEON effizienter, nach weiteren Optimierungen evtl. auch schneller
\end{itemize}
\chapter{Zusammenfassung und Ausblick}
\label{ch:sum}
\rm

\section{Zusammenfassung}\label{sec:summary}
Am Institut f�r Mikroelektronische Systeme der Leibniz Universit�t Hannover werden in einem Teilbereich die die Umsetzungsm�glichkeiten einer Musikklassifikation auf mobilen Endger�ten untersucht. Hierf�r werden Explorationen auf verschiedenen Prozessorachitekturen durch gef�hrt, mit dem Ziel einen Entwurfsraum zu definieren.\\\\
Im Rahmen dieser Arbeit wurden drei verschiedene Architekturen aus diesem Entwurfsraum auf ihre Einsetzbarkeit zur Musikklassifikation untersucht. Diese drei Architekturen wurden durch einen ARM Cortex A8, einen C674x DSP und einer heterogenen Architektur aus der Kombination aus beiden repr�sentiert. Diese sollten optimiert und auf Basis ihrer Laufzeit und Energieeffizenz bewertet werden.\\
Untersuchungen sollten hierf�r auf Basis von vier vorgegebenen Musikklassifikationsmethoden durchgef�hrt werden. Es stellte sich heraus, dass der zeitkritischste Schritt der Musikklassifikation in der Merkmalsextraktion bestand, daher wurden Optimierungen auch nur in diese Richtung durchgef�hrt. Nach der Optimierung wurden die Extraktionen auf dem Cortex A8 um Faktoren zwischen 2,5 (MCL3) und 22,5 (MCL4) je nach Methode beschleunigt. Auf dem C674x DSP wurden Beschleunigungen um Faktoren zwischen 4,2 (MCL3) und 10,5 (MCL4) erreicht. Hierbei muss erw�hnt werden, dass Extraktionen auf dem DSP im Allgemeinen schneller ausgef�hrt wurden, so dass die Extraktion von zum Beispiel MCL1 in den optimiertesten Versionen um einen Faktor von 1,8 schneller durchgef�hrt wird. Es scheint also, dass der DSP effektiver ist. Hierbei darf aber nicht au�er acht gelassen werden, dass der DSP nicht arbeiten kann, ohne dass der ARM zumindest im Idle-Modus ist. Somit hat der DSP einen h�heren Energieverbrauch und ist damit wieder ineffizienter. Eine Variante in der Prozessierung und Klassifikation ebenfalls auf dem DSP ausgef�hrt werden, wurde nicht untersucht.\\
F�r eine Kombination aus Cortex A8 und C674x wurde nur der sequenzielle Fall untersucht, so dass der DSP die Extraktion und danach der Cortex A8 Prozessierung und Klassifikation ausf�hrt. Auf Basis dieser Untersuchung wurde festgestellt, dass diese heterogene Architektur aus ARM und DSP ineffizienter ist, als wenn alle drei Schritte auf dem ARM ausgef�hrt werden, auch wenn der heterogene Ansatz schneller w�re.\\
Abschlie�end ist das Ergebnis dieser Arbeit, dass alle drei Architekturen f�r den Einsatz in einem mobilen Endger�t geeignet w�ren. Hierbei stellte sich der ARM als am energieeffizientesten heraus.

\section{Ausblick}\label{sec:vista}

Innerhalb dieser Arbeit wurden einige Aspekte der drei betrachteten Architekturen nicht betrachtet. Weiteres Forschungen sollten daher in Richtung des parallelen Einsatzes von ARM und DSP gehen. Hierbei sollte eine Verwendung des DSP zur Merkmalsextraktion angestrebt werden, bei der der DSP mehrere Musikst�cke verarbeitet und danach der ARM diese Prozessiert und Klassifiziert. In dieser Zeit sollte der DSP weitere Lieder verarbeiten.\\
Des weiteren wurden auch nicht alle Optimierungsm�glichkeiten in vollem Ausma� ausgesch�pft, so dass auch in diese Richtung weitere Forschung geschehen sollte.

%bibtex Referenzen in bib datei
\nocite{*}
\bibliography{References/references}
\bibliographystyle{Macros/unsrtdin}
%\bibliographystyle{unsrt}
%Literaturverzeichnis ins Inhaltsverzeichnis
\addcontentsline{toc}{chapter}{Literaturverzeichnis}

\appendix %%Anhang
\renewcommand{\chaptermark}[1]{\markright{Appendix \thechapter\ \it #1}}
\renewcommand{\sectionmark}[1]{\markright{Appendix \thechapter\ \it #1}}
\renewcommand{\appendixname}{Anhang}
\renewcommand{\tablename}{Table}
\renewcommand{\figurename}{Figure}
\chapter{}
\rm

\section{Tabellen}
In diesem Kapitel sind Tabellen mit zus�tzlichen Informationen zu Sachverhalten des Hauptteils aufgef�hrt. Die Reihenfolge entspricht den Verweisen im Hauptteil.
\begin{table}[h]
\centering
\begin{tabular}{|c|c|p{6cm}|}
\hline
Befehl & Takte & Kommentar\\
\hline\hline
ADD & 1 &\\
CMP & 1 &\\
LDM & 2-n & n = Register/2, min 2 Cycles\\  
LDR & 1-2 & Je nach Offset\\
MOV \& MOVN & 1 &\\
MUL \& MLA & 1-3 & Je nach Art\\
STM & 2-n & n = Register/2, min 2 Cycles\\  
STR & 1-2 & Je nach Offset\\
SUB & 1 &\\
MCR & min. 60 & h�ngt von der Auslastung des angesprochenen Coprozessors ab\\
MRC & min. 60 & h�ngt von der Auslastung des angesprochenen Coprozessors ab\\
\hline
\end{tabular}
\caption{Auszug der Integeroprationen des ARMv7\cite{cortexa8}}
\label{tab:inst}
\end{table}

\input{Misc/clearpage}

\end{document}

