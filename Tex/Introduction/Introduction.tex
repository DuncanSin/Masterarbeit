\chapter{Einf�hrung}
\label{ch:intro}
\rm

Im Spektrum der Forschungsgebiete der digitalen Signalverarbeitung hat sich die Untersuchung von Architekturen und Algorithmen zur Audiosignal-Klassifikationssystemen als sehr interessant herausgestellt. Hierbei geht es um die Untersuchung der Umsetzbarkeit von hardwareunterst�tzten Audiosignal-Klassifikation auf mobilen Endger�ten.\\
Da mobile Endger�te aber nur begrenzte Ressourcen zur Verf�gung stellen k�nnen, wie z.B. Akkuleistung und Speichervolumen, ist es n�tig eine Prozessorarchtektur zu finden, auf der eine Audiosignal-Klassifikation mit optimalem Ressourcenverbrauch verarbeitet werden kann.
Aus diesem Grund wird eine Exploration des Entwurfraums von Hardware-Architekturen durchgef�hrt um eine solche Prozessorarchitektur zu identifizieren.\\
Im Rahmen dieser Arbeit werden drei solche Architekturen auf in Hinblick auf ihre Laufzeit und Energieeffizienz untersucht. Diese drei Architekturen beinhalten einen ARM Cortex A8 der Firma ARM Ltd., einen C674x DSP der Firma Texas Instruments und die heterogene Prozessorarchitektur eines DaVinci\texttrademark-Prozessors ebenfalls von der Firma Texas Instruments, der aus den beiden vorhergenannten Prozessoren besteht.\\
Zu diesem Zweck wird in \textbf{Kapitel \ref{ch:grund}} als erstes eine Wiederholung der grundlegenden Eigenschaften der Musikklassifikation, Merkmalen und der verwendeten Musikklassifikationsmethoden gegeben. \textbf{Kapitel \ref{ch:board}} befasst sich daraufhin mit der Beschreibung der Architekturen von ARM Cortex A8 und C674x DSP. Hierbei werden ebenfalls das verwendete Entwicklungsboard EVM8168 der Firma Texas Instruments und die verwendete Entwicklungsumgebung vorgestellt. In \textbf{Kapitel \ref{ch:opt}} werden anschlie�end die Implementation und Optimierung auf den beiden Prozessorachritekturen des ARM Cortex A8 und des C674x DSP vorgestellt und erl�utert. Eine Evaluation der extrahierten Daten wird in \textbf{Kapitel \ref{ch:results}} durchgef�hrt, bevor am Schluss in \textbf{Kapitel \ref{ch:sum}} eine Zusammenfassung der in dieser Arbeit beschriebenen und gefundenen Erkenntnisse gemacht und ein Ausblick f�r weitere Forschung gegeben wird.  