\chapter{Einf�hrung}
\label{ch:intro}
\rm

Im Spektrum der Forschungsgebiete der digitalen Signalverarbeitung hat sich die Untersuchung von Architekturen und Algorithmen von Audiosignal-Klassifikationssystemen als sehr interessant herausgestellt. Hierbei geht es um die Untersuchung der Umsetzbarkeit von hardwareunterst�tzter Audiosignal-Klassifikation auf mobilen Endger�ten.\\
Da mobile Endger�te aber nur begrenzte Ressourcen zur Verf�gung stellen k�nnen, wie z.B. Akkuleistung und Speichervolumen, ist es n�tig eine Prozessorarchtektur zu finden, auf der eine Audiosignal-Klassifikation mit optimalem Ressourcenverbrauch verarbeitet werden kann.
Aus diesem Grund wird eine Exploration des Entwurfraums von Hardware-Architekturen durchgef�hrt um eine solche Prozessorarchitektur zu identifizieren.\\
Musikklassifikation wird in drei Schritten ausgef�hrt: Merkmalsextraktion, Prozessierung extrahierter Daten und die anschlie�ende Klassifizierung der Musikst�cke auf Basis dieser Daten. Typische Verfahren zur Musikklassifikation weisen einen hohen Rechenaufwand im Extraktionsschritt auf, so dass besonderes Augenmerk auf der energieeffizienten diese Berechnung gelegt werden muss.\\
Im Rahmen dieser Arbeit soll eine dieser Prozessorarchitekturen auf ihren energieeffiziente Einsetzbarkeit f�r die mobile Umsetzung der Audiosignal-Klassifikation (nachfolgend Musikklassifikation genannt) untersucht werden. Bei dieser Architektur handelt es sich um eine Heterogene Prozessorarchitektur, die durch einen DaVinci Video Prozessor der Firma Texas Instruments repr�sentiert wird. Dieser Prozessor besteht aus einem ARM Cortex-A8 und einem TI C674x DSP. F�r die Untersuchung dieser Architektur ist eine Optimierung des Extraktionsschrittes auf beiden Prozessorkernen durchzuf�hren, um hinterher den Einsatz einer gleichzeitigen Benutzung beider Prozessorkerne zur Musikklassifikation bewerten zu k�nnen.\\
In \textbf{Kapitel \ref{ch:grund}} werden daher vorab die Grundlagen der Musikklassifikation inklusive der ben�tigten Algorithmen zur Merkmalsextraktion, Prozessierung und Klassifikation vorgestellt. Des weiteren werden in diesem Kapitel ebenfalls die verwendete Musikklassifikationsmethoden erl�utert, die f�r ein Profiling und eine Bewertung herangezogen werden sollen. Daraufhin wird in \textbf{Kapitel \ref{ch:board}} das f�r die Untersuchungen verwendete Evaluationsboard EVM8168 und die sich darauf befindliche heterogene Prozessorarchitektur des DaVinci Video Prozessors n�her vorgestellt. Au�erdem wird in diesem Kapitel die verwendete Entwicklungsumgebung EZSDK vorgestellt, die f�r die Entwicklung von Programmen auf diesem Prozessor von der Firma Texas Instruments zur Verf�gung gestellt wird. In \textbf{Kapitel \ref{ch:opt}} wird anschlie�end die Implementierung der Musikklassifikation, die Optimierungsstrategie und die auf Basis dieser durchgef�hrten Optimierungen vorgestellt und bewertet. Eine Evaluation der gesammelten Ergebnisse wird in \textbf{Kapitel \ref{ch:results}} durch gef�hrt. Abschlie�end wird in \textbf{Kapitel \ref{ch:sum}} eine Zusammenfassung der Erkenntnisse dieser Arbeit und ein Ausblick f�r weitere Forschung gegeben. 